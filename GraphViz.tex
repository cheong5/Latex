\documentclass{oblivoir}
\usepackage{graphicx}


\title{객체지향의 특성, 상속에 대해 알아보자 }
\author{Stud. Son Cheong-sol, Dept. of Computer Science}

\begin{document}
\maketitle

많은 언어들이 발전 해온 가운데 가장 각광받는 언어, 객체지향 언어들이 현재 시대를 대표하고 있다. 그렇다면 이번에는 객체지향의 대표적인 특징인 추상화, 상속, 다형성들 중에서 상속에 대해 알아보도록 하자. \\
아래 그림을 보자.

\begin{figure}[h]
  \centering
  \includegraphics[width=1\textwidth]{inheritance1.pdf}
  \caption{ people, students, professor가 가지는 속성들.}
\end{figure}

people, students, professor 모두가 age와 name이라는 속성을 가지고 있다. 중복되는 속성을 보니 왠지 중복을 피하고 싶은 욕구가 솟구친다. 한번 students와 professor의 속성에서 중복되는 두가지 속성을 제거해보자

\begin{figure}[h]
  \centering
  \includegraphics[width=0.7\textwidth]{inheritance2.pdf}
  \caption{ students, professor에서 중복되는 속성을 제거.}
\end{figure}

뭔가 깔끔해 지긴 했는데 결과가 이상해 졌다. students와 professor가 id만 가지고 있어서 원래 가지고 있던 속성들, age와 name이 있었는지, 없었는지 알 수가 없는 것이다. 그렇다면 students와 professor가 각각 people의 속성들을 '상속'받게 되면 어떨까?\\
아래의 그림을 보도록 하자.

\begin{figure}[h]
  \centering
  \includegraphics[width=0.5\textwidth]{inheritance3.pdf}
  \caption{ students, professor에게 people의 속성을 상속.}
\end{figure}

이렇게 선으로 연결됨으로 students와 professor 모두가 people의 하위항목으로 들어가며 age와 name이라는 속성또한 가지게 됨을 확인 할 수 있다. 이때 가지고 있던 속성을 제공하는 class를 super, bass, parent class라고 하며 속성을 상속받는 class를 sub, derived, child class라고 한다.

\end{document} 
