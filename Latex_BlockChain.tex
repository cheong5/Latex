\documentclass[a4paper]{oblivoir}
\usepackage{tabu}
\usepackage{array}
\usepackage{makecell}
\title{ 블록체인에 대해 막연한 지식을 자세하게 만들어 보자}
\author{손청솔/컴퓨터공학전공\\
    부산대학교}


\begin{document}
\maketitle

\section{ 블록체인 }

컴퓨터를 조금 다룬다 하는 사람들은 물론이고 컴퓨터에 대해 문외한이라고 생각하는 사람들도 이젠 블록체인이라라는 단어가 익숙할 시기이다. 여러 매체나 신문에서 블로체인이 어쩐다 저쩐다 하는 이야기를 질리도록 들었기 때문이다. 하지만 그렇게 블록체인이라는 것을 알고 있는 사람들 중에 블록체인이 어떤것이고 어떻게 어떤분야에 사용되냐고 물어보면 대답할 수 있는 사람이 몇명이나 될까? 그중 10퍼센트의 사람들이라도 제대로 답 할 수 있을까? 나도 물론 제대로 대답할 수 없는 대다수의 사람들 중 한명이다. 하지만 나는 컴퓨터공학을 전공하는 학생으로서 대답할 수 있어야 한다는 생각에 블록체인에 대해 한번 제대로 살펴볼까 한다.


\section{ 블록체인의 탄생 }
블록체인은 사실 가상화폐를 다루기 위해 더욱 안전하고 투명한 거래시스템을 위해 고안된 시스템이다. 여러 분야에서 새로운 기술에 사용되고있지만 실상은 가상거래를 더욱 견고하게 만든 시스템인 것이다. \\ 처음 이 블록체인이라는 개념을 제안한 사은 사토시 나카모토(Satoshi Nakamoto)라는 가명을 쓰는 사람이었다. 2008년 10월 31일에 공개된 그의 논문 '비트코인: 개인 대 개인의 전자화폐 시스템\cite{im}'에서 블록체인 기술이 적용된 암호화폐 '비트코인(bitcoin)'을 개발하고 C++언어로 작성된 코드를 배포했다.

\section{ 블록체인이 무엇인가 } \label{burbon}
쉽게 말해서 전자화폐를 사용하는 내역들, 다시말해 거래 장부를 보관하는 특별한 자료구조이다. P2P방식으로 중앙처리 시스템이 필요 없으며, 모든 유저들이 내역을 열람할 수 있지만 누구도 수정은 할 수 없는, 투명하고 견고한 자료구조인 것이다.\\
그럼 조금 자세히 살펴보자. 먼저 블록체인에서는 A가 B에게 돈을 주면 B는 A에게서 돈을 받는, 2행위를 거래의 최소단위로 트랜잭션이라고 한다. 그리고 블록체인은 블록을 하나의 단위로 이루어지는데 이 블록에는 트랜잭션별로 hash함수를 사용하여 함호화되어 저장된다. 저장된 hash값들을 다시 hash함수로 묶어 하나의 hash, root hash로 만들어 저장하는데 하나의 블록은 반드시 하나의 root hash를 가져야 한다.\\
이렇게 새로운 블록을 이미 존재하는 블록들, 기존의 블록체인에 연결하려면 블록의 이름을 정해주어야 하는데 블록의 이름을 정하는데에 소요되는 시간과 자원이 어마어마하다.\\
블록이 완성되면 시스템에 연결된 모든 유저들에게 완성됐다는 사실이 전파되는데 각 유저들은 이 블록의 유횽성을 과반수의 유저가 블록의 생성을 승인할 경우 블록은 정식으로 블록체인에 연결된다. 만일 서로 다른 거래내역을 가진 블록이 동시에 생성될 경우 시스템은 약 1시간 정도 기다리며 지켜보다 가장 긴 길이의 블록체인을 정식 블록체인으로 선택하고 선택되지 못한 블록들은 버려지게 된다.

\section{ 어떻게 사용할까 }\label{conclusions}
블록체인을 사용하고자 하는 연구는 정말 활발히 진행되고 있다. 우선 블록체인의 탄생에 큰 영향을 끼친 가상화폐\cite{coin}부터 시작해 은행\cite{bank}, 멀티미디어\cite{media}, 가상화폐 관련 법\cite{law}분야까지 정말 두루두루 연구되고 응용하려 한다. \\
하지만 이 블록체인이라는 새로운 개념에대해 의문을 제기하는 사람들도 있다.\\ 블록체인이 뭐길래 쓰자고 하는거지 기존에 사용해도 전혀 문제 없었던 공인 인증서를 쓰거나 비밀번호를 더 자주 바꾸면 되지. \\ 눈에 보이지도 않고 실제로 존재하는 것도 아닌데 가상화폐는 너무 위험해.\\
가상화폐 사용자들이 직접 화폐를 발행하면 은행의 기능이 자연스럽게 약해질텐데 내가 투자한 돈이나 은행 종사자들은 직장을 잃게 되겠어.\\
많은 사람들이 생각하고 걱정하는 문제들은 당연히 제기되어야 하는 문제들이다. 심지어 컴퓨터가 만들어내는 음악을 우리는 어떻게 받아들여야 하는걸까. \\
이렇게 많은 문제들이 예상되는 상황에 제 4차 산업혁명의 중심이라고 평가되는 블록체인 기술은 조심스럽게 논의 될 필요가 있는 부분이다.\\
다음 표는 중국과 국제조합의 블록체인 기술 관련 표준화 과정을 비교한 것이다. 

\begin{center}

\begin{tabular}{ |>{\centering}m{1.5cm}|>{\centering}m{2cm}|>{\centering}m{2cm}|>{\centering}m{2cm}| }
  \hline
  \tabularnewline
  구분 & 2015 & 2016 & 2017  \tabularnewline
  \hline
  국제 & - & 오스트리아 표준 협회 전혀 새로운 국가 표준화 방안 제시, ISO 제출(4월) & ISO/IC표준기술위원회 1차 회의에서 업무팀과 연구팀 설치(4월)  \tabularnewline
  \hline
  중국 & 블록체인 연구연맹, 블록체인 응용연구센터 발족(12월) & 중관촌 블록체인산업연맹발족(2월) & 블록체인 기술 기반 디지털어음거래 플렛트폼 시험가동 성공(2월) \tabularnewline
  \hline
\end{tabular}
\end{center}
\begin{center}
\centering{표1 블록체인 기술 관련 표준화 과정 비교\cite{CSF}}
\end{center}
\section{결론}
세계는 변하고 있다. 블록체인, 가상화폐, 최신 network기술에 대해 전혀 관심이 없는 사람들 조차 블록체인은 알고있다. 잘을 몰라도 그 이름은 분명히 들었을 것이다. 우리나라가 한 기술의 선두를 달리고 있는 분야가 몇이나 될까?\\
새로운 분야가 이토록 세계적인 주목을 받고있다면 우리도 앞다투어 달려가야 하는 것은 아닐까? 앞장서진 않더라도 도채되지는 않게, 정말 관심없는 분야라도 모르쇠로 일관하지 않도록 세계에 뒤쳐지지 않는 경쟁력을 가질 수 있도록 조금더 세계의 동향을 주목하고 함께 나아갈 준비를 지금부터 해도록 하자.

\bibliographystyle{plain}
\bibliography{Survey}
\end{document} 
